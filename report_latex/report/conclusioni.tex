\chapter{Conclusions}
\label{conclusioni}
\thispagestyle{empty}

\noindent For this Data Mining project activity we were presented with a simulated seismic data mining problem and the task to predict the time remaining until the next laboratory earthquake; starting from raw data, we pre-processed it by computing aggregated representative data to be fed to a model of our choice. We tested different models from the most used Python library, \texttt{scikit-learn}, starting from the most simple ones up until neural networks, and we encountered various issues that we tried to dig deep into with the intention of understanding them and drawing our conclusions. All things considered, it looks like more complex models tend to overfit to the training data, giving a good score on the training set and a much worse one on the test set, while simpler models seem to be more general, giving unsatisfactory results on the training set but performing much better on the test set.

Further development of the presented models could be accomplished by tweaking and refining the parameters settings, a task that we did not pursue due to time restrictions, knowledge boundaries, and to our common decision to focus more on exploring different models and collecting the responses.